\documentclass[9pt , a4paper]{article}
 \usepackage[utf8]{inputenc}
 \usepackage[french]{babel}
 \usepackage{amssymb}  %pour utiliser \blacksquare
 \usepackage{amsmath}
 \usepackage{amsthm} % theoreme
 \usepackage[hmargin=2cm,vmargin=3.5cm]{geometry}
 \usepackage{float} % force une figure de se mettre au bon endroit
 \usepackage{graphicx}
 \usepackage{color}
 \usepackage{enumerate}
 \usepackage[svgnames]{xcolor}
\usepackage{listings}
 \usepackage{pict2e} % permet de dessiner 
 %\usepackage{hyperref} % permet de rajouter des liens dans le pdf
 \usepackage{fancyhdr} % relatif au pied de page
 \pagestyle{fancy} % relatif au pied de page
 \usepackage{lastpage} % numérotation de page
 \usepackage{textcomp} % copyleft
 \usepackage{wrapfig} % figure
 \usepackage{lscape} %format paysage
 \definecolor{gris}{rgb}{0.95,0.95,0.95}
 \usepackage{rotating}
 \graphicspath{{ImagesForLatex/}}
 \usepackage{xcolor}
\usepackage{colortbl,hhline}
\usepackage{subfig}
\usepackage{url}
 	
	\lstset{numbers=left, tabsize=4, backgroundcolor=\color{gris},
frame=single, breaklines=true,
keywordstyle=\color{black},
stringstyle=\ttfamily,
framexleftmargin=6mm, xleftmargin=6mm}
 %tabular
 \newcolumntype{M}[1]{>{\raggedright}m{#1}}
 	
 % headers et footers
 \lhead{LINGI1113}
 \rhead{Projet minix}
 \cfoot{\thepage\ sur \pageref{LastPage}}
	
 % definition du couleurs
\xdefinecolor{azure}{named}{Azure}
\xdefinecolor{gray}{named}{Gray}
  
 
% Redéfinition du premier niveau
\renewcommand{\theenumi}{{(}\alph{enumi}{)}}
\renewcommand{\labelenumi}{\theenumi}
 
% Redéfinition du deuxiéme niveau (bloc itemize)
\renewcommand{\theenumii}{\Alph{enumii}}
\renewcommand{\labelenumii}{\theenumii}


  
%\paragraph*{Citation}  	  	 
% \begin{quotation}
%   $\prec$
%
%   $\succ$
% \end{quotation}
% Prenom Nom - - - \textsl{titre}

\begin{document}
  
  	\begin{titlepage}
		\begin{center}
			{\huge \textsc{Projet Minix}}\\
			\vspace{0.4cm}
			
			{\Large {Professeur : xxx \\ \vspace{0.2cm} Assistants : Christophe Paasch et Fabien Duchêne }}\\
			\vspace{0.6cm}
			
			{\Large \textit{ LINGI1113: Système Informatique 2}}\\
			\vspace{1.2cm}

			\texttt{}\\
			\vspace{0.2cm}
			\vspace{0.1cm}
			{\Large \textbf{Universit\'e Catholique de Louvain}}
			\vspace{0.7cm}

			\vspace{0.2cm}

			Martin \textsc{Donies} \\
			Florentin \textsc{Rochet} \\
			\vspace{0.2cm}
			2011-2012\\
		\end{center}
	\end{titlepage}

	\newpage
	
	\section{Introduction}
	
	Nous avons réalisé correctement les appels systèmes correspondant aux ressources suivantes : \\
	
	\begin{itemize}
		\item RLIMIT NICE
		\item RLIMIT NPROC
		\item RLIMIT NOFILE
		\item RLIMIT FSIZE \\
	\end{itemize}
	
	Pour faciliter nos explications et votre compréhension de notre Minix modifié, nous allons lister et expliquer dans ce rapport  les éléments du patch généré par la commande make dist.
	
	\section{Architecture globale}
	
	\section{listing et explications des modifications}
	
\end{document}